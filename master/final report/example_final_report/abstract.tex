\begin{abstract}
Surface integration is an important step for automatic 3D reconstruction of real objects. The goal of a surface integration algorithm is to reconstruct a surface from a set of range images registered in a common coordinate system. Based on the surface representation used, existing algorithms can be divided into two categories: volume-based and mesh-based. Volume-based methods have been shown to be robust to scanner noise and small features (regions of high curvature) and can build water tight models of high quality. It is, however, difficult to choose the appropriate voxel size when the input range images have both small features and large registration errors compared to the sampling density of range images. Mesh-based methods are more efficient and need less memory compared to volume-based methods but these methods fail in the presence of small features and are not robust to scanning noise. \\

This paper presents a robust algorithm for mesh-based surface integration of a set of range images. The algorithm is incremental and operates on a range image and the model reconstructed so far. Our algorithm first, transform the model in the coordinate system of the range image. Then, it finds the regions of model overlapping with the range image. This is done by shooting rays from the scanner, through the vertices in the range image and intersecting them with the model. Finally, the algorithm integrates the overlapping regions by using weighted average of points in the model and the  range image. The weights are computed using the scanner uncertainty and helps in reducing the effects of scanning noise. To handle small features robustly the integration of overlapping regions is done by computing the position of vertices in the range image along the scanner's line of sight. Since for every point in a range image there is exactly one depth value, the reconstructed surface in the regions of high curvature will not have self-intersections.
\end{abstract}